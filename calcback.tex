% Created 2018-09-25 Di 22:18
% Intended LaTeX compiler: pdflatex
\documentclass[11pt]{article}
\usepackage[utf8]{inputenc}
\usepackage[T1]{fontenc}
\usepackage{graphicx}
\usepackage{grffile}
\usepackage{longtable}
\usepackage{wrapfig}
\usepackage{rotating}
\usepackage[normalem]{ulem}
\usepackage{amsmath}
\usepackage{textcomp}
\usepackage{amssymb}
\usepackage{capt-of}
\usepackage{hyperref}
\author{Nikolai Weidt}
\date{\today}
\title{Calcback Debug}
\hypersetup{
 pdfauthor={Nikolai Weidt},
 pdftitle={Calcback Debug},
 pdfkeywords={},
 pdfsubject={},
 pdfcreator={Emacs 26.1 (Org mode 9.1.14)}, 
 pdflang={English}}
\begin{document}

\maketitle
\tableofcontents


\section{What is this?}
\label{sec:org780831c}
Im coding a programm to get the complex refractive index \(n = n * ik\) from the ellipsometric parameters \(\Delta\) and \(\Psi\) I got from a simulation.
\section{Imports:}
\label{sec:orgbf785b2}
\begin{verbatim}
import numpy as np
import matplotlib
from matplotlib import pyplot
\end{verbatim}

\section{Defining some variables:}
\label{sec:org04ef2e2}
Defining some variables for later use:

\begin{verbatim}
CSVFILE = "300nmSiO2.csv"
phi_i = 70 * np.pi / 180
d_L = 300
n_air = 1
rerange = 5
imrange = 5
\end{verbatim}

\section{Read .csv-file:}
\label{sec:org3476096}
Read the values into a two dimensional numpy array as [[lambda,Psi,Delta,T,R,n\(_{\text{S,k}}\)\(_{\text{S}}\)],[],[]\ldots{}]

\begin{verbatim}
csv = np.loadtxt(CSVFILE, delimiter=",", skiprows=1)
\end{verbatim}

\begin{verbatim}
[[3.00000000e+02 5.52217535e+01 8.43722832e+01 ... 2.23744631e-01
  2.67260000e+00 3.03750000e+00]
 [3.03000000e+02 5.01118744e+01 9.33085011e+01 ... 3.19674321e-01
  2.73460000e+00 3.03810000e+00]
 [3.06000000e+02 4.63582455e+01 9.84368139e+01 ... 4.08053172e-01
  2.79670000e+00 3.03680000e+00]
 ...
 [2.39400000e+03 3.10615656e+01 1.04467643e+02 ... 4.27263440e-01
  3.40030000e+00 0.00000000e+00]
 [2.39700000e+03 3.10226710e+01 1.04442500e+02 ... 4.27940008e-01
  3.40020000e+00 0.00000000e+00]
 [2.40000000e+03 3.09839244e+01 1.04417327e+02 ... 4.28613566e-01
  3.40010000e+00 0.00000000e+00]]
\end{verbatim}

\section{Calculate \(\rho\)}
\label{sec:org10139ac}
\subsection{Create a matrix containing every possible refractive index (n+ik):}
\label{sec:org0ab67dd}
\begin{verbatim}
lsp_re = np.linspace(0.1, rerange, 1001)
lsp_im = np.linspace(0.1, imrange, 1001)
re, im = np.meshgrid (lsp_re, lsp_im, copy=False)
matrix = 1j * im + re
\end{verbatim}

\begin{verbatim}
print(matrix)
\end{verbatim}

\begin{verbatim}
[[0.1   +0.1j    0.1049+0.1j    0.1098+0.1j    ... 4.9902+0.1j
  4.9951+0.1j    5.    +0.1j   ]
 [0.1   +0.1049j 0.1049+0.1049j 0.1098+0.1049j ... 4.9902+0.1049j
  4.9951+0.1049j 5.    +0.1049j]
 [0.1   +0.1098j 0.1049+0.1098j 0.1098+0.1098j ... 4.9902+0.1098j
  4.9951+0.1098j 5.    +0.1098j]
 ...
 [0.1   +4.9902j 0.1049+4.9902j 0.1098+4.9902j ... 4.9902+4.9902j
  4.9951+4.9902j 5.    +4.9902j]
 [0.1   +4.9951j 0.1049+4.9951j 0.1098+4.9951j ... 4.9902+4.9951j
  4.9951+4.9951j 5.    +4.9951j]
 [0.1   +5.j     0.1049+5.j     0.1098+5.j     ... 4.9902+5.j
  4.9951+5.j     5.    +5.j    ]]
\end{verbatim}
\subsection{Snell's Law:}
\label{sec:org58f36ce}
\begin{verbatim}
phi_L = np.arcsin(np.sin(phi_i)*n_air)/ n_L) 
phi_L = np.arcsin(np.sin(phi_L)*n_L)/ n_S) 
\end{verbatim}

\subsection{Fresnel Equations:}
\label{sec:org890aceb}
\subsection{Calculate rho (Fujiwara):}
\label{sec:org91b0e2e}
\end{document}